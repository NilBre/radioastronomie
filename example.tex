% This example is meant to be compiled with lualatex or xelatex
% The theme itself also supports pdflatex
\PassOptionsToPackage{unicode}{hyperref}
\documentclass[aspectratio=1610, 9pt]{beamer}

% Load packages you need here
\usepackage{polyglossia}
\setmainlanguage{german}

\usepackage{csquotes}
    

\usepackage{amsmath}
\usepackage{amssymb}
\usepackage{mathtools}

\usepackage{hyperref}
\usepackage{bookmark}

% load the theme after all packages

\usetheme[
  showtotalframes, % show total number of frames in the footline
]{tudo}

% Put settings here, like
\unimathsetup{
  math-style=ISO,
  bold-style=ISO,
  nabla=upright,
  partial=upright,
  mathrm=sym,
}

%Titel:
\title{Radiolaute aktive Galaxienkerne}
%Autor
\author[N.Breer]{Nils Breer}
%Lehrstuhl/Fakultät
\institute{Fakultät Physik}
%Titelgrafik muss ich einfueren!!!
%\titlegraphic{\includegraphics[width=0.3\textwidth]{content/Bilder/interferenz.jpg}}
\date{28.11.2019}

\begin{document}
\maketitle

\begin{frame}\frametitle{Agenda}
  \begin{itemize}
    \item Was ist ein AGN \"uberhaupt?
    \item Historie und Entdeckung der AGN
    \item Aufbau und Klassifizierung von AGN
    \item Detektion von AGN
    \item Warum sind AGN nun so interessant?
  \end{itemize}
\end{frame}

\begin{frame}\frametitle{Was ist ein AGN?}
  \begin{itemize}
    \item Zentralregion von Galaxien
    \item enorme Strahlungsdeposition
  \end{itemize}
\end{frame}

\begin{frame}\frametitle{Historie und Entdeckung}
  \begin{itemize}
    \item 1909, Fath: Anormales Spektrum von Spiralnebeln; unerwartete Emissionslinien von ionisiertem Gas
    \item 1943, Seyfert: helle, sternartige Kerne mit breiten Emissionslinien; au\ss erdem spektroskopische Betrachtungen
    \item 1954: Cygnus A und Virgo A starke Radioquellen; Emission bis $10^{38}$W
    \item 1960er,Quasar 3C 273: Emissionslinie stark verschoben $\to$ gro\ss e Rotverschiebung
  \end{itemize}
\end{frame}

% bild einfuegen

\begin{frame}\frametitle{Aufbau und Klassifizierung von AGN}
  \begin{block}{Aufbau und Kriterien}
  \begin{columns}
  \begin{column}[c]{0.45\textwidth}
  \begin{itemize}
    \item Staubtorus
    \item SMBH
    \item Akkretions-Scheibe
    \item Jets
    \item breites Emissionsspektrum an nicht-thermischer Strahlung
    \item schnelle Variabilit\"at
    \item kleines Emissionsgebiet (Jetorigin) ~ 7 AE
    \item schnelle Rotation von Gas um Zentrum ~ einige AE
    \item hohe Geschwindigkeiten $\to$ hohe Masse %(M87 SMBH)
  \end{itemize}
  \end{column}
  \begin{column}[c]{0.45\textwidth}
    \includegraphics{images/agn-pic.png}
  \end{column}
  \end{columns}
  \end{block}
\end{frame}

\begin{frame}\frametitle{Aufbau und Klassifizierung von AGN}
  \begin{block}{Klassifizierung}
    \begin{itemize}
      \item Seyfert-Galaxien
      \item Radiogalaxien
      \item Blazare
      \item Fanaroff-Riley Galaxien
    \end{itemize}
  \end{block}
\end{frame}


\begin{frame}\frametitle{Seyfert-Galaxien}
  \begin{itemize}
    \item radioleiser Galaxientyp
    \item ca. 90 \% sind Spiralgalaxien
    \item h\"aufigster Galaxientyp mit AGN
    \item Heller, blauer Kern
    \item Emissionslinie f\"uhrt auf hei\"sses ionisiertes Gas
    \item Helligkeits\"anderung des Kerns schnell (f\"ur astronomische Zeitskalen)
    \item Hoher Geschindigkeit des Gases im Zentrum der Galaxie
  \end{itemize}
\end{frame}

\begin{frame}\frametitle{Fanaroff-Riley Galaxien}
  \begin{block}{FR Typ 1}
  \begin{columns}
  \begin{column}{0.45\textwidth}
    \begin{itemize}
      \item Radioemission \~ 1/r
      \item h\"aufig Back-to-Back Jets
    \end{itemize}
  \end{column}
  \begin{column}{0.45\textwidth}
    \includegraphics{images/FR1.png}
  \end{column}
  \end{columns}
  \end{block}
\end{frame}

\begin{frame}\frametitle{Fanaroff-Riley Galaxien}
  \begin{block}{FR Typ 2}
  \begin{columns}
  \begin{column}[c]{0.45\linewidth}
    \begin{itemize}
      \item Helle "Hot Spots" in sog. Lobes
      \item relativ dunkler Kern
      \item oft einseitiger Jet
      \item i.A. heller als FR-1 durch Keulen
    \end{itemize}
  \end{column}
  \begin{column}[c]{0.45\linewidth}
    \includegraphics{images/FR2.png}
      % hier ein bild von FR1 1 einfuegen
  \end{column}
  \end{columns}
  \end{block}
\end{frame}

\begin{frame}\frametitle{Blazare}
  \begin{block}{Typen}
  \begin{columns}
  \begin{column}[c]{0.45\linewidth}
    \begin{itemize}
      \item BL Lac-Quasar
      \item Doppelh\"ocker Spektrum
      \item AGN mit Blick in Jetachse
      \item bis zu 50 \% der Galaxienmasse in Form von Strahlung
      \begin{itemize}
        \item ver\"anderlicher Stern (Cuo Hoffmeister, Sternbild Eidechse)
      \end{itemize}
    \end{itemize}
    \end{column}
    \begin{column}[c]{0.45\linewidth}
      \includegraphics{images/bl-lac-object.png}
    \end{column}
    \end{columns}
  \end{block}
\end{frame}

\begin{frame}\frametitle{Jet Modelle}
  \begin{columns}
  \begin{column}[c]{0.45\linewidth}
  \begin{itemize}
    \item Synchrotron-Self-Compton Modell
    \begin{itemize}
      \item Synchrotron Emission durch $e^{-}$, first Bump
      \item HE Bump: inverser Compton Effekt mit eigenen Photonen
      \item h\"aufig f\"ur BL Lacertae Spektren
    \end{itemize}
    \item External Compton Modell
    \begin{itemize}
      \item Synchrotron Emission durch $e^{-}$, first Bump
      \item HE Bump: inverser Compton Effekt mit eigenen Photonen
    \end{itemize}
  \end{itemize}
  \end{column}
  \begin{column}{0.45\linewidth}
    \begin{block}{Proton-Blazar Modell}
    \begin{itemize}
      \item HE Bump: $\gamma$ aus $\pi^{0}$ Zerf\"allen
      \item Synchro. von Protonen
      \item Synchro. von Myonen, $\pi^{\pm}$ Zerf\"alle
    \end{itemize}
    \end{block}
  \end{column}
  \end{columns}
\end{frame}

\begin{frame}\frametitle{Detektionsm\"oglichkeiten}
  \begin{itemize}
    \item VLBI (very large baseline interferometry)
    \begin{itemize}
      \item gute Winkelaufl\"osung
      \item Continuum VLBI Surveys
    \end{itemize}
    \item polarimetric observations
    \begin{itemize}
      \item 
    \end{itemize}
  \end{itemize}
\end{frame}

\begin{frame}\frametitle{Warum sind radiolaute AGN nun interessant?}
  \begin{columns}
  \begin{column}[c]{0.45\textwidth}
  \begin{itemize}
    \item Radiospektren sind auf der Erde gut sichtbar.
    \item Radiowellen werden weniger stark absorbiert von galaktischem Staub als Licht
    \item Untersuchung von Pulsaren, SNR, galaktische Zentrum der Milchstra\ss e
    \item Bezugspunkte im Universum
  \end{itemize}
  \end{column}
  \begin{column}[c]{0.45\textwidth}
    \includegraphics{images/durchlaessigkeit.png}
  \end{column}
  \end{columns}
\end{frame}

\begin{frame}
Quellen: \\
\url{https://www.slideshare.net/antglezatienza/galaxias-activas-87180963} \\
\url{https://ned.ipac.caltech.edu/level5/Urry1/UrryP4_3.html} \\
\url{http://ned.ipac.caltech.edu/level5/Glossary/Essay_fanaroff.html} \\

\end{frame}

\end{document}
